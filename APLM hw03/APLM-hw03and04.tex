\documentclass[]{article}
\usepackage{lmodern}
\usepackage{amssymb,amsmath}
\usepackage{ifxetex,ifluatex}
\usepackage{fixltx2e} % provides \textsubscript
\ifnum 0\ifxetex 1\fi\ifluatex 1\fi=0 % if pdftex
  \usepackage[T1]{fontenc}
  \usepackage[utf8]{inputenc}
\else % if luatex or xelatex
  \ifxetex
    \usepackage{mathspec}
  \else
    \usepackage{fontspec}
  \fi
  \defaultfontfeatures{Ligatures=TeX,Scale=MatchLowercase}
\fi
% use upquote if available, for straight quotes in verbatim environments
\IfFileExists{upquote.sty}{\usepackage{upquote}}{}
% use microtype if available
\IfFileExists{microtype.sty}{%
\usepackage[]{microtype}
\UseMicrotypeSet[protrusion]{basicmath} % disable protrusion for tt fonts
}{}
\PassOptionsToPackage{hyphens}{url} % url is loaded by hyperref
\usepackage[unicode=true]{hyperref}
\hypersetup{
            pdftitle={APLM hw03and04},
            pdfauthor={san teng},
            pdfborder={0 0 0},
            breaklinks=true}
\urlstyle{same}  % don't use monospace font for urls
\usepackage[margin=1in]{geometry}
\usepackage{color}
\usepackage{fancyvrb}
\newcommand{\VerbBar}{|}
\newcommand{\VERB}{\Verb[commandchars=\\\{\}]}
\DefineVerbatimEnvironment{Highlighting}{Verbatim}{commandchars=\\\{\}}
% Add ',fontsize=\small' for more characters per line
\usepackage{framed}
\definecolor{shadecolor}{RGB}{248,248,248}
\newenvironment{Shaded}{\begin{snugshade}}{\end{snugshade}}
\newcommand{\KeywordTok}[1]{\textcolor[rgb]{0.13,0.29,0.53}{\textbf{#1}}}
\newcommand{\DataTypeTok}[1]{\textcolor[rgb]{0.13,0.29,0.53}{#1}}
\newcommand{\DecValTok}[1]{\textcolor[rgb]{0.00,0.00,0.81}{#1}}
\newcommand{\BaseNTok}[1]{\textcolor[rgb]{0.00,0.00,0.81}{#1}}
\newcommand{\FloatTok}[1]{\textcolor[rgb]{0.00,0.00,0.81}{#1}}
\newcommand{\ConstantTok}[1]{\textcolor[rgb]{0.00,0.00,0.00}{#1}}
\newcommand{\CharTok}[1]{\textcolor[rgb]{0.31,0.60,0.02}{#1}}
\newcommand{\SpecialCharTok}[1]{\textcolor[rgb]{0.00,0.00,0.00}{#1}}
\newcommand{\StringTok}[1]{\textcolor[rgb]{0.31,0.60,0.02}{#1}}
\newcommand{\VerbatimStringTok}[1]{\textcolor[rgb]{0.31,0.60,0.02}{#1}}
\newcommand{\SpecialStringTok}[1]{\textcolor[rgb]{0.31,0.60,0.02}{#1}}
\newcommand{\ImportTok}[1]{#1}
\newcommand{\CommentTok}[1]{\textcolor[rgb]{0.56,0.35,0.01}{\textit{#1}}}
\newcommand{\DocumentationTok}[1]{\textcolor[rgb]{0.56,0.35,0.01}{\textbf{\textit{#1}}}}
\newcommand{\AnnotationTok}[1]{\textcolor[rgb]{0.56,0.35,0.01}{\textbf{\textit{#1}}}}
\newcommand{\CommentVarTok}[1]{\textcolor[rgb]{0.56,0.35,0.01}{\textbf{\textit{#1}}}}
\newcommand{\OtherTok}[1]{\textcolor[rgb]{0.56,0.35,0.01}{#1}}
\newcommand{\FunctionTok}[1]{\textcolor[rgb]{0.00,0.00,0.00}{#1}}
\newcommand{\VariableTok}[1]{\textcolor[rgb]{0.00,0.00,0.00}{#1}}
\newcommand{\ControlFlowTok}[1]{\textcolor[rgb]{0.13,0.29,0.53}{\textbf{#1}}}
\newcommand{\OperatorTok}[1]{\textcolor[rgb]{0.81,0.36,0.00}{\textbf{#1}}}
\newcommand{\BuiltInTok}[1]{#1}
\newcommand{\ExtensionTok}[1]{#1}
\newcommand{\PreprocessorTok}[1]{\textcolor[rgb]{0.56,0.35,0.01}{\textit{#1}}}
\newcommand{\AttributeTok}[1]{\textcolor[rgb]{0.77,0.63,0.00}{#1}}
\newcommand{\RegionMarkerTok}[1]{#1}
\newcommand{\InformationTok}[1]{\textcolor[rgb]{0.56,0.35,0.01}{\textbf{\textit{#1}}}}
\newcommand{\WarningTok}[1]{\textcolor[rgb]{0.56,0.35,0.01}{\textbf{\textit{#1}}}}
\newcommand{\AlertTok}[1]{\textcolor[rgb]{0.94,0.16,0.16}{#1}}
\newcommand{\ErrorTok}[1]{\textcolor[rgb]{0.64,0.00,0.00}{\textbf{#1}}}
\newcommand{\NormalTok}[1]{#1}
\usepackage{graphicx,grffile}
\makeatletter
\def\maxwidth{\ifdim\Gin@nat@width>\linewidth\linewidth\else\Gin@nat@width\fi}
\def\maxheight{\ifdim\Gin@nat@height>\textheight\textheight\else\Gin@nat@height\fi}
\makeatother
% Scale images if necessary, so that they will not overflow the page
% margins by default, and it is still possible to overwrite the defaults
% using explicit options in \includegraphics[width, height, ...]{}
\setkeys{Gin}{width=\maxwidth,height=\maxheight,keepaspectratio}
\IfFileExists{parskip.sty}{%
\usepackage{parskip}
}{% else
\setlength{\parindent}{0pt}
\setlength{\parskip}{6pt plus 2pt minus 1pt}
}
\setlength{\emergencystretch}{3em}  % prevent overfull lines
\providecommand{\tightlist}{%
  \setlength{\itemsep}{0pt}\setlength{\parskip}{0pt}}
\setcounter{secnumdepth}{0}
% Redefines (sub)paragraphs to behave more like sections
\ifx\paragraph\undefined\else
\let\oldparagraph\paragraph
\renewcommand{\paragraph}[1]{\oldparagraph{#1}\mbox{}}
\fi
\ifx\subparagraph\undefined\else
\let\oldsubparagraph\subparagraph
\renewcommand{\subparagraph}[1]{\oldsubparagraph{#1}\mbox{}}
\fi

% set default figure placement to htbp
\makeatletter
\def\fps@figure{htbp}
\makeatother


\title{APLM hw03and04}
\author{san teng}
\date{2018/12/26}

\begin{document}
\maketitle

\begin{Shaded}
\begin{Highlighting}[]
\KeywordTok{library}\NormalTok{(stringr)}
\KeywordTok{setwd}\NormalTok{(}\StringTok{"D:/github_oicjacky/Applied Statistical Linear Model/APLM hw03"}\NormalTok{)}
\KeywordTok{rm}\NormalTok{(}\DataTypeTok{list =} \KeywordTok{ls}\NormalTok{())}
\NormalTok{data <-}\StringTok{ }\KeywordTok{read.table}\NormalTok{(}\StringTok{"CH09TA01.txt"}\NormalTok{)}
\KeywordTok{colnames}\NormalTok{(data) <-}\StringTok{ }\KeywordTok{c}\NormalTok{(}\StringTok{"X1"}\NormalTok{,}\StringTok{"X2"}\NormalTok{,}\StringTok{"X3"}\NormalTok{,}\StringTok{"X4"}\NormalTok{,}\StringTok{"X5"}\NormalTok{,}\StringTok{"Y"}\NormalTok{,}\StringTok{"lnY"}\NormalTok{)}

\NormalTok{n <-}\StringTok{ }\KeywordTok{dim}\NormalTok{(data)[}\DecValTok{1}\NormalTok{]}
\NormalTok{P <-}\StringTok{ }\DecValTok{5}            \CommentTok{# 5 predict variable}
\end{Highlighting}
\end{Shaded}

\paragraph{power set}\label{power-set}

\begin{Shaded}
\begin{Highlighting}[]
\NormalTok{powerset =}\StringTok{ }\ControlFlowTok{function}\NormalTok{(s)\{}
\NormalTok{  len =}\StringTok{ }\KeywordTok{length}\NormalTok{(s)}
\NormalTok{  l =}\StringTok{ }\KeywordTok{vector}\NormalTok{(}\DataTypeTok{mode=}\StringTok{"list"}\NormalTok{,}\DataTypeTok{length=}\DecValTok{2}\OperatorTok{^}\NormalTok{len) }
\NormalTok{  l[[}\DecValTok{1}\NormalTok{]]=}\KeywordTok{numeric}\NormalTok{()}
\NormalTok{  counter =}\StringTok{ }\DecValTok{1}
  \ControlFlowTok{for}\NormalTok{(x }\ControlFlowTok{in} \DecValTok{1}\OperatorTok{:}\KeywordTok{length}\NormalTok{(s))\{}
    \ControlFlowTok{for}\NormalTok{(subset }\ControlFlowTok{in} \DecValTok{1}\OperatorTok{:}\NormalTok{counter)\{}
\NormalTok{      counter=counter}\OperatorTok{+}\DecValTok{1}
\NormalTok{      l[[counter]] =}\StringTok{ }\KeywordTok{c}\NormalTok{(l[[subset]],s[x])}
\NormalTok{    \}}
\NormalTok{  \}}
  \KeywordTok{return}\NormalTok{(l)}
\NormalTok{\}}
\CommentTok{#powerset(1:P)}
\end{Highlighting}
\end{Shaded}

\subsection{\texorpdfstring{\(R^2\) \& adjusted
\(R^2\)}{R\^{}2 \& adjusted R\^{}2}}\label{r2-adjusted-r2}

\begin{Shaded}
\begin{Highlighting}[]
\CommentTok{# R squares}
\NormalTok{R.square <-}\StringTok{ }\ControlFlowTok{function}\NormalTok{(data , mod)\{}
  
\NormalTok{  SSTO <-}\StringTok{ }\KeywordTok{sum}\NormalTok{( ( data}\OperatorTok{$}\NormalTok{Y }\OperatorTok{-}\StringTok{ }\KeywordTok{mean}\NormalTok{(data}\OperatorTok{$}\NormalTok{Y) )}\OperatorTok{^}\DecValTok{2}\NormalTok{ )}
\NormalTok{  SSE <-}\StringTok{ }\KeywordTok{sum}\NormalTok{(mod}\OperatorTok{$}\NormalTok{residuals}\OperatorTok{^}\DecValTok{2}\NormalTok{)}
  
  \KeywordTok{return}\NormalTok{( }\DecValTok{1} \OperatorTok{-}\StringTok{ }\NormalTok{(SSE }\OperatorTok{/}\StringTok{ }\NormalTok{SSTO) )}
\NormalTok{\}}
\CommentTok{# adjusted R squares}
\NormalTok{R.adjust <-}\StringTok{ }\ControlFlowTok{function}\NormalTok{(data , mod)\{}
\NormalTok{  n <-}\StringTok{ }\KeywordTok{dim}\NormalTok{(data)[}\DecValTok{1}\NormalTok{]}
\NormalTok{  p <-}\StringTok{ }\NormalTok{mod}\OperatorTok{$}\NormalTok{rank}
  
\NormalTok{  SSTO <-}\StringTok{ }\KeywordTok{sum}\NormalTok{( ( data}\OperatorTok{$}\NormalTok{Y }\OperatorTok{-}\StringTok{ }\KeywordTok{mean}\NormalTok{(data}\OperatorTok{$}\NormalTok{Y) )}\OperatorTok{^}\DecValTok{2}\NormalTok{ )}
\NormalTok{  SSE <-}\StringTok{ }\KeywordTok{sum}\NormalTok{(mod}\OperatorTok{$}\NormalTok{residuals}\OperatorTok{^}\DecValTok{2}\NormalTok{)}
  
  \KeywordTok{return}\NormalTok{( }\DecValTok{1} \OperatorTok{-}\StringTok{ }\NormalTok{( (n}\OperatorTok{-}\DecValTok{1}\NormalTok{) }\OperatorTok{/}\StringTok{ }\NormalTok{(n}\OperatorTok{-}\NormalTok{p) ) }\OperatorTok{*}\StringTok{ }\NormalTok{(SSE }\OperatorTok{/}\StringTok{ }\NormalTok{SSTO) )}
\NormalTok{\}}
\end{Highlighting}
\end{Shaded}

\begin{Shaded}
\begin{Highlighting}[]
\CommentTok{#length(powerset(1:P))}
\NormalTok{all_possible <-}\StringTok{ }\KeywordTok{powerset}\NormalTok{(}\DecValTok{1}\OperatorTok{:}\NormalTok{P)}
\NormalTok{R_adj.square <-}\StringTok{ }\NormalTok{R_square <-}\StringTok{ }\KeywordTok{c}\NormalTok{()}
\NormalTok{variable <-}\StringTok{ }\KeywordTok{c}\NormalTok{()}
\ControlFlowTok{for}\NormalTok{(i }\ControlFlowTok{in} \DecValTok{2}\OperatorTok{:}\StringTok{ }\KeywordTok{length}\NormalTok{(all_possible) ) \{}
  
  \ControlFlowTok{if}\NormalTok{( }\KeywordTok{length}\NormalTok{(all_possible[[i]]) }\OperatorTok{==}\StringTok{ }\DecValTok{1}\NormalTok{ )\{}
\NormalTok{    A <-}\StringTok{ }\KeywordTok{data.frame}\NormalTok{( data[, all_possible[[i]] ] , }
                     \DataTypeTok{Y =}\NormalTok{ data}\OperatorTok{$}\NormalTok{Y                         )}
    \KeywordTok{colnames}\NormalTok{(A)[}\OperatorTok{-}\DecValTok{2}\NormalTok{] <-}\StringTok{ }\KeywordTok{paste0}\NormalTok{(}\StringTok{"X"}\NormalTok{,all_possible[[i]])}
\NormalTok{    a <-}\StringTok{ }\KeywordTok{R.adjust}\NormalTok{(A , }\KeywordTok{lm}\NormalTok{( Y }\OperatorTok{~}\StringTok{ }\NormalTok{A[,}\DecValTok{1}\NormalTok{] ,A) )}
\NormalTok{    b <-}\StringTok{ }\KeywordTok{R.square}\NormalTok{(A , }\KeywordTok{lm}\NormalTok{( Y }\OperatorTok{~}\StringTok{ }\NormalTok{A[,}\DecValTok{1}\NormalTok{] ,A) )}
    \CommentTok{#print(colnames(A))}
\NormalTok{  \}}\ControlFlowTok{else} \ControlFlowTok{if}\NormalTok{( }\KeywordTok{length}\NormalTok{(all_possible[[i]]) }\OperatorTok{==}\StringTok{ }\DecValTok{2}\NormalTok{ )\{}
\NormalTok{    A <-}\StringTok{ }\KeywordTok{data.frame}\NormalTok{( data[, all_possible[[i]] ] [ ,}\DecValTok{1}\NormalTok{] , }
\NormalTok{                     data[, all_possible[[i]] ] [ ,}\DecValTok{2}\NormalTok{] ,}
                     \DataTypeTok{Y =}\NormalTok{ data}\OperatorTok{$}\NormalTok{Y                         )}
    \KeywordTok{colnames}\NormalTok{(A)[}\OperatorTok{-}\DecValTok{3}\NormalTok{] <-}\StringTok{ }\KeywordTok{colnames}\NormalTok{(data[, all_possible[[i]] ])}
\NormalTok{    a <-}\StringTok{ }\KeywordTok{R.adjust}\NormalTok{(A , }\KeywordTok{lm}\NormalTok{( Y }\OperatorTok{~}\StringTok{ }\NormalTok{A[,}\DecValTok{1}\NormalTok{] }\OperatorTok{+}\StringTok{ }\NormalTok{A[,}\DecValTok{2}\NormalTok{] ,A) )}
\NormalTok{    b <-}\StringTok{ }\KeywordTok{R.square}\NormalTok{(A , }\KeywordTok{lm}\NormalTok{( Y }\OperatorTok{~}\StringTok{ }\NormalTok{A[,}\DecValTok{1}\NormalTok{] }\OperatorTok{+}\StringTok{ }\NormalTok{A[,}\DecValTok{2}\NormalTok{] ,A) )}
    \CommentTok{#print(colnames(A))}
    
\NormalTok{  \}}\ControlFlowTok{else} \ControlFlowTok{if}\NormalTok{( }\KeywordTok{length}\NormalTok{(all_possible[[i]]) }\OperatorTok{==}\StringTok{ }\DecValTok{3}\NormalTok{ )\{}
\NormalTok{    A <-}\StringTok{ }\KeywordTok{data.frame}\NormalTok{( data[, all_possible[[i]] ] [ ,}\DecValTok{1}\NormalTok{] , }
\NormalTok{                     data[, all_possible[[i]] ] [ ,}\DecValTok{2}\NormalTok{] ,}
\NormalTok{                     data[, all_possible[[i]] ] [ ,}\DecValTok{3}\NormalTok{] ,}
                     \DataTypeTok{Y =}\NormalTok{ data}\OperatorTok{$}\NormalTok{Y                         )}
    \KeywordTok{colnames}\NormalTok{(A)[}\OperatorTok{-}\DecValTok{4}\NormalTok{] <-}\StringTok{ }\KeywordTok{colnames}\NormalTok{(data[, all_possible[[i]] ])}
\NormalTok{    a <-}\StringTok{ }\KeywordTok{R.adjust}\NormalTok{(A , }\KeywordTok{lm}\NormalTok{( Y }\OperatorTok{~}\StringTok{ }\NormalTok{A[,}\DecValTok{1}\NormalTok{] }\OperatorTok{+}\StringTok{ }\NormalTok{A[,}\DecValTok{2}\NormalTok{] }\OperatorTok{+}\StringTok{ }\NormalTok{A[,}\DecValTok{3}\NormalTok{] ,A) )}
\NormalTok{    b <-}\StringTok{ }\KeywordTok{R.square}\NormalTok{(A , }\KeywordTok{lm}\NormalTok{( Y }\OperatorTok{~}\StringTok{ }\NormalTok{A[,}\DecValTok{1}\NormalTok{] }\OperatorTok{+}\StringTok{ }\NormalTok{A[,}\DecValTok{2}\NormalTok{] }\OperatorTok{+}\StringTok{ }\NormalTok{A[,}\DecValTok{3}\NormalTok{] ,A) )}
    \CommentTok{#print(colnames(A))}
    
\NormalTok{  \}}\ControlFlowTok{else} \ControlFlowTok{if}\NormalTok{( }\KeywordTok{length}\NormalTok{(all_possible[[i]]) }\OperatorTok{==}\StringTok{ }\DecValTok{4}\NormalTok{ )\{}
\NormalTok{    A <-}\StringTok{ }\KeywordTok{data.frame}\NormalTok{( data[, all_possible[[i]] ] [ ,}\DecValTok{1}\NormalTok{] , }
\NormalTok{                     data[, all_possible[[i]] ] [ ,}\DecValTok{2}\NormalTok{] ,}
\NormalTok{                     data[, all_possible[[i]] ] [ ,}\DecValTok{3}\NormalTok{] ,}
\NormalTok{                     data[, all_possible[[i]] ] [ ,}\DecValTok{4}\NormalTok{] ,}
                     \DataTypeTok{Y =}\NormalTok{ data}\OperatorTok{$}\NormalTok{Y                         )}
    \KeywordTok{colnames}\NormalTok{(A)[}\OperatorTok{-}\DecValTok{5}\NormalTok{] <-}\StringTok{ }\KeywordTok{colnames}\NormalTok{(data[, all_possible[[i]] ])}
\NormalTok{    a <-}\StringTok{ }\KeywordTok{R.adjust}\NormalTok{(A , }\KeywordTok{lm}\NormalTok{( Y }\OperatorTok{~}\StringTok{ }\NormalTok{A[,}\DecValTok{1}\NormalTok{] }\OperatorTok{+}\StringTok{ }\NormalTok{A[,}\DecValTok{2}\NormalTok{] }\OperatorTok{+}\StringTok{ }\NormalTok{A[,}\DecValTok{3}\NormalTok{] }\OperatorTok{+}\StringTok{ }\NormalTok{A[,}\DecValTok{4}\NormalTok{] ,A) )}
\NormalTok{    b <-}\StringTok{ }\KeywordTok{R.square}\NormalTok{(A , }\KeywordTok{lm}\NormalTok{( Y }\OperatorTok{~}\StringTok{ }\NormalTok{A[,}\DecValTok{1}\NormalTok{] }\OperatorTok{+}\StringTok{ }\NormalTok{A[,}\DecValTok{2}\NormalTok{] }\OperatorTok{+}\StringTok{ }\NormalTok{A[,}\DecValTok{3}\NormalTok{] }\OperatorTok{+}\StringTok{ }\NormalTok{A[,}\DecValTok{4}\NormalTok{] ,A) )}
    \CommentTok{#print(colnames(A))}
    
\NormalTok{  \}}\ControlFlowTok{else} \ControlFlowTok{if}\NormalTok{( }\KeywordTok{length}\NormalTok{(all_possible[[i]]) }\OperatorTok{==}\StringTok{ }\DecValTok{5}\NormalTok{ )\{}
\NormalTok{    A <-}\StringTok{ }\KeywordTok{data.frame}\NormalTok{( data[, all_possible[[i]] ] [ ,}\DecValTok{1}\NormalTok{] , }
\NormalTok{                     data[, all_possible[[i]] ] [ ,}\DecValTok{2}\NormalTok{] ,}
\NormalTok{                     data[, all_possible[[i]] ] [ ,}\DecValTok{3}\NormalTok{] ,}
\NormalTok{                     data[, all_possible[[i]] ] [ ,}\DecValTok{4}\NormalTok{] ,}
\NormalTok{                     data[, all_possible[[i]] ] [ ,}\DecValTok{5}\NormalTok{] ,}
                     \DataTypeTok{Y =}\NormalTok{ data}\OperatorTok{$}\NormalTok{Y                         )}
    \KeywordTok{colnames}\NormalTok{(A)[}\OperatorTok{-}\DecValTok{6}\NormalTok{] <-}\StringTok{ }\KeywordTok{colnames}\NormalTok{(data[, all_possible[[i]] ])}
\NormalTok{    a <-}\StringTok{ }\KeywordTok{R.adjust}\NormalTok{(A , }\KeywordTok{lm}\NormalTok{( Y }\OperatorTok{~}\StringTok{ }\NormalTok{A[,}\DecValTok{1}\NormalTok{] }\OperatorTok{+}\StringTok{ }\NormalTok{A[,}\DecValTok{2}\NormalTok{] }\OperatorTok{+}\StringTok{ }\NormalTok{A[,}\DecValTok{3}\NormalTok{] }\OperatorTok{+}\StringTok{ }\NormalTok{A[,}\DecValTok{4}\NormalTok{] }\OperatorTok{+}\StringTok{ }\NormalTok{A[,}\DecValTok{5}\NormalTok{] ,A) )}
\NormalTok{    b <-}\StringTok{ }\KeywordTok{R.square}\NormalTok{(A , }\KeywordTok{lm}\NormalTok{( Y }\OperatorTok{~}\StringTok{ }\NormalTok{A[,}\DecValTok{1}\NormalTok{] }\OperatorTok{+}\StringTok{ }\NormalTok{A[,}\DecValTok{2}\NormalTok{] }\OperatorTok{+}\StringTok{ }\NormalTok{A[,}\DecValTok{3}\NormalTok{] }\OperatorTok{+}\StringTok{ }\NormalTok{A[,}\DecValTok{4}\NormalTok{] }\OperatorTok{+}\StringTok{ }\NormalTok{A[,}\DecValTok{5}\NormalTok{] ,A) )}
    \CommentTok{#print(colnames(A))}
\NormalTok{  \}}
  
\NormalTok{  R_adj.square <-}\StringTok{ }\KeywordTok{rbind}\NormalTok{(R_adj.square , a )}
\NormalTok{  R_square <-}\StringTok{ }\KeywordTok{rbind}\NormalTok{(R_square , b )}
\NormalTok{  variable <-}\StringTok{ }\KeywordTok{c}\NormalTok{(variable ,}\KeywordTok{str_c}\NormalTok{(}\KeywordTok{colnames}\NormalTok{(A)[}\OperatorTok{-}\KeywordTok{dim}\NormalTok{(A)[}\DecValTok{2}\NormalTok{]] ,}\DataTypeTok{collapse =} \StringTok{","}\NormalTok{))}
\NormalTok{\}}
\NormalTok{A <-}\StringTok{ }\KeywordTok{data.frame}\NormalTok{(}\DataTypeTok{R_square =}\NormalTok{ R_square ,}
                \DataTypeTok{R_adj.square =}\NormalTok{ R_adj.square ,}
                \DataTypeTok{variable =}\NormalTok{ variable )}
\end{Highlighting}
\end{Shaded}

\paragraph{\texorpdfstring{the model with highest
\(R^2\)}{the model with highest R\^{}2}}\label{the-model-with-highest-r2}

\begin{verbatim}
##       R_square R_adj.square       variable
## b.30 0.6949877    0.6632155 X1,X2,X3,X4,X5
\end{verbatim}

\paragraph{\texorpdfstring{the model with highest adjusted
\(R^2\)}{the model with highest adjusted R\^{}2}}\label{the-model-with-highest-adjusted-r2}

\begin{verbatim}
##       R_square R_adj.square    variable
## b.22 0.6910939    0.6658771 X1,X2,X3,X5
\end{verbatim}

\subsection{CV(1) or leave-one-out cross
validation}\label{cv1-or-leave-one-out-cross-validation}

\begin{Shaded}
\begin{Highlighting}[]
\NormalTok{candidate <-}\StringTok{ }\KeywordTok{powerset}\NormalTok{(}\DecValTok{1}\OperatorTok{:}\NormalTok{P)[}\OperatorTok{-}\DecValTok{1}\NormalTok{]}

\NormalTok{CV_value <-}\StringTok{ }\KeywordTok{rep}\NormalTok{(}\DecValTok{0}\NormalTok{ ,}\KeywordTok{length}\NormalTok{(candidate) )}
\ControlFlowTok{for}\NormalTok{ (j }\ControlFlowTok{in} \DecValTok{1}\OperatorTok{:}\KeywordTok{length}\NormalTok{(candidate) ) \{}
  
\NormalTok{  pred.value <-}\StringTok{ }\KeywordTok{rep}\NormalTok{(}\DecValTok{0}\NormalTok{ ,n)}
  \ControlFlowTok{for}\NormalTok{(i }\ControlFlowTok{in} \DecValTok{1}\OperatorTok{:}\NormalTok{n)\{}
\NormalTok{    d <-}\StringTok{ }\KeywordTok{combn}\NormalTok{(}\DecValTok{1}\OperatorTok{:}\NormalTok{n ,}\DecValTok{1}\NormalTok{)[,i] }\CommentTok{# CV(1) or leave-one-out cross validation}
\NormalTok{    data_train <-}\StringTok{ }\NormalTok{data[}\OperatorTok{-}\NormalTok{d ,]}
\NormalTok{    data_test <-}\StringTok{ }\NormalTok{data[ d ,]}
    \CommentTok{# training}
\NormalTok{    xnam <-}\StringTok{ }\KeywordTok{paste0}\NormalTok{(}\StringTok{"X"}\NormalTok{, candidate [[j]] )}
\NormalTok{    fmla <-}\StringTok{ }\KeywordTok{as.formula}\NormalTok{(}\KeywordTok{paste}\NormalTok{(}\StringTok{"Y ~ "}\NormalTok{, }\KeywordTok{paste}\NormalTok{(xnam, }\DataTypeTok{collapse=} \StringTok{"+"}\NormalTok{)))}
\NormalTok{    model <-}\StringTok{ }\KeywordTok{lm}\NormalTok{(fmla ,data_train) }
    \CommentTok{# testing}
\NormalTok{    pred.value[i] <-}\StringTok{ }\KeywordTok{as.numeric}\NormalTok{(}
\NormalTok{      (data_test}\OperatorTok{$}\NormalTok{Y }\OperatorTok{-}\StringTok{ }\KeywordTok{as.matrix}\NormalTok{(}\KeywordTok{cbind}\NormalTok{(}\DecValTok{1}\NormalTok{ ,data_test[,xnam])) }\OperatorTok\StringTok{ }\NormalTok{model}\OperatorTok{$}\NormalTok{coefficients)}\OperatorTok{^}\DecValTok{2}\NormalTok{ )}
    
\NormalTok{  \}}
\NormalTok{  CV_value[j] <-}\StringTok{ }\KeywordTok{sum}\NormalTok{(pred.value) }\OperatorTok{/}\StringTok{ }\KeywordTok{dim}\NormalTok{(}\KeywordTok{combn}\NormalTok{(}\DecValTok{1}\OperatorTok{:}\NormalTok{n ,}\DecValTok{1}\NormalTok{))[}\DecValTok{2}\NormalTok{]}
  \CommentTok{#print(xnam)}
\NormalTok{\}}
\end{Highlighting}
\end{Shaded}

\paragraph{the model with smallest CV(1)
value}\label{the-model-with-smallest-cv1-value}

\begin{verbatim}
## [1] "X1,X2,X3"
\end{verbatim}

\end{document}
